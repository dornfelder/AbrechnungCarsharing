%%%%%%%%%%%%%%%%%%%%%%%%%%%%%%%%%%%%%%%%%
% Elmueller Formal Letter
% LaTeX Template
% Version 1.3 (20/11/16)
%
% This template has been downloaded from:
% http://www.LaTeXTemplates.com
%
% Original author:
% Micha Elmueller (http://micha.elmueller.net/) with modifications by 
% Vel (vel@LaTeXTemplates.com)
%
% License:
% CC BY-NC-SA 3.0 (http://creativecommons.org/licenses/by-nc-sa/3.0/)
%
%%%%%%%%%%%%%%%%%%%%%%%%%%%%%%%%%%%%%%%%%

%----------------------------------------------------------------------------------------
%	DOCUMENT CONFIGURATIONS
%----------------------------------------------------------------------------------------

\documentclass[
	pagenumber=false, % Removes page numbers from page 2 onwards when false
	parskip=half, % Separates paragraphs with some whitespace, use parskip=full for more space or comment out to return to default
	fromalign=right, % Aligns the from address to the right
	foldmarks=true, % Prints small fold marks on the left of the page
	addrfield=true % Set to false to hide the addressee section - you will then want to adjust the height of the body of the letter on the page by adding the following in this section: \makeatletter \@setplength{refvpos}{\useplength{toaddrvpos}} \makeatletter
	firstfoot=on % Add footer
    ]{scrlttr2}

\usepackage[utf8]{inputenc} %enable german umlaute and ß
\usepackage[T1]{fontenc} % For extra glyphs (accents, etc)
\usepackage{stix} % Use the Stix font by default
\usepackage[german]{babel} % Explicitly load the babel package to stop an error occurring on some LaTeX installations
\usepackage{fancyhdr}
\usepackage[official]{eurosym}
\usepackage{hhline}


\renewcommand*{\raggedsignature}{\raggedright} % Stop the signature from indenting

\fancypagestyle{plain}{
\renewcommand{\headrulewidth}{0pt} 
\fancyhf{}
\fancyfoot[L]{%
    \usekomavar{frombank}%
    }
\fancyfoot[R]{%
    Seite \thepage%
    }
}
\pagestyle{plain}

\makeatletter
\let\ps@empty\ps@plain
\let\ps@firstpage\ps@plain
\makeatother


%----------------------------------------------------------------------------------------
%	YOUR INFORMATION AND LETTER DATE
%----------------------------------------------------------------------------------------

\setkomavar{fromname}{CSA Aurich e. V., Torsten Hoffmann} % Your name used in the from address

\setkomavar{fromaddress}{Lehmweg 4, 26603 Aurich} % Your address

\setkomavar{signature}{Alexandra Seregely} % Your name used in the signature

\setkomavar{frombank}{Car Sharing Aurich e.V.; Amtsgericht Aurich Registerblatt VR 681\\
Bankverbindung: Raiffeisen Volksbank eG; BLZ: 285 622 97; Konto: 444 101 000}

\date{ 18 August 2017 } % Date of the letter

%----------------------------------------------------------------------------------------
 
\begin{document}

%----------------------------------------------------------------------------------------
%	ADDRESSEE
%----------------------------------------------------------------------------------------
 
\begin{letter}{ Alexandra Seregely \\ Eickebuscher Weg 29\\ 26603 Aurich } % Addressee name and address

%----------------------------------------------------------------------------------------
%	LETTER CONTENT
%---------------------------------------------------------------------------------------
\setkomavar{subject}{\bf{Monatsabrechnung Juni 2017}}

\opening{Moin,}

im oben genannten Zeitraum haben Sie im Car Sharing Aurich e.V. als Mitglied die folgende(n) Fahrt(en) zurückgelegt.

Dabei ist der nachfolgend aufgeführte Aufwand entstanden, welcher von Ihrem Konto eingezogen wird.

\closing{Mit freundlichen Grüßen}

%\setkomavar*{enclseparator}{Attached} % Change the default "encl:" to "Attached:"
%\encl{Copyright permission form} % Attached documents


%----------------------------------------------------------------------------------------

\end{letter}


\begin{tabular}{ | c | c | c | c | c | c | c | c | c | }
\hline
Datum & Fahrzeug & Dauer [h] & Berechnete Kilometer [km]& Kosten [\euro{}]\\ \hline
04 06 2017&aurCs108&34.0&24&25.0\\\hline
07 06 2017&aurCs108&10.0&27&13.5\\
\hhline{|-|-|-|-|=|}
\multicolumn{4}{|r|}{Summe:} &38.5\\\hline
\end{tabular}


\end{document}
